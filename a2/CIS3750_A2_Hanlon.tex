\documentclass[12pt,letterpaper]{article}
\usepackage[utf8]{inputenc}
\usepackage{kpfonts}
\usepackage[T1]{fontenc}

% custom titles
\usepackage{titlesec}

% fix broken title numbering with 2016 titlesec update
\usepackage{etoolbox}

\makeatletter
\patchcmd{\ttlh@hang}{\parindent\z@}{\parindent\z@\leavevmode}{}{}
\patchcmd{\ttlh@hang}{\noindent}{}{}{}
\makeatother
%%%

\titlespacing*\section{0pt}{12pt plus 4pt minus 2pt}{0pt plus 2pt minus 2pt}
\titlespacing*\subsection{0pt}{12pt plus 4pt minus 2pt}{0pt plus 2pt minus 2pt}
\titlespacing*\subsubsection{0pt}{12pt plus 4pt minus 2pt}{0pt plus 2pt minus 2pt}

% Package for double spacing
\usepackage{setspace}
\usepackage{ragged2e}

% Set 1.0 inch margins
\usepackage[margin=1.0in, headheight=15pt]{geometry}
\usepackage{enumitem}

% Use images and graphics
\usepackage{graphicx}
\usepackage{float}

% Use nicer headers
\usepackage{fancyhdr}
\pagestyle{fancy}
\renewcommand{\headrulewidth}{0pt}
\rhead{CIS3750 - Assignment 1}

% sections should be indexed with alphabets
\renewcommand{\thesection}{\Alph{section}}

%double spaced lines in the whole document
\doublespacing

\title{Assignment 1}

\begin{document}
\begin{titlepage}
    \centering
    \vspace*{\baselineskip}
    \rule{\textwidth}{1.6pt}\vspace*{-\baselineskip}\vspace*{2pt}
    \rule{\textwidth}{0.4pt}\\[1.5\baselineskip]
    {\LARGE \textsc{An Initial Outline of a Software System to Improve Food Security in Malawi}}\\[\baselineskip]
	\rule{\textwidth}{0.4pt}\vspace*{-\baselineskip}\vspace{4pt}    
    \rule{\textwidth}{2pt}\\[2\baselineskip]
   
    \vspace*{5\baselineskip}
    \textsc{BY}\\[0.25\baselineskip]
    {\LARGE HANLON} \\
    
    \vspace*{\baselineskip}
    % List of authors in alphabetical order (by last name)
    {\textsc{David DiMaria \\ Braydon Johnson \\ Joshua Lemieux \\ Neivin Mathew \\ Like Zheng} \par}
    \vfill
    {\scshape October 14, 2016} \\
  \end{titlepage}
  
  
% Table of Contents (no page numbers on contents)
\pagenumbering{roman} %roman numerals for ToC
\tableofcontents
\lhead{} % remove default header from Contents page
\clearpage
\pagenumbering{arabic} %pagenumbering in arabic numbers
    
\section{Client Details}
Malawi is a country located in the warm heart of Africa, with a population of 16.4 million. According to the Malawi Vulnerability Assessment Committee, an estimated 2.83 million people will experience acute food insecurity during the 2015/16 lean season. (World Food Programme, 2016) The economy of the country is based on agriculture. The majority of the people in Malawi are farmers who cultivate tobacco for living. Due to the impact of climate change, decline in global tobacco consumption and the scarcity of food in Malawi, the farmers need to transition to growing different crops. However, since Malawian farmers have been growing tobacco for many generations, the current farmers of the country lack the tools and knowledge required in order to start growing other crops. 

The client for the project is the Agricultural Research and Extension Trust (ARET) of Malawi. They are the premier research institution of Malawi, and are responsible for conducting research and providing technical/extension services on tobacco. The trust was established on September 1, 1995 to foster development and information dissemination for Malawi's tobacco industry. It combined the services of two institutions, the Tobacco Research Institute of Malawi (TRIM) and the Estate Extension Service Trust (EEST), who separately provided research and extension services respectively. (ARET, 2016)

ARET states their new vision is "to be a leading regional centre of excellence in agricultural research and technology dissemination which promotes diversification in the agricultural sector." (ARET, 2016). To accomplish this, they require a software system to collect data on the farmers in Malawi, and distribute research conducted within the organization to the farmers of the country. This project aims to fulfil this need.

\clearpage
\section{Team Details}
\subsection{Team Name}
The team name for the project is "Hanlon."\par
The name is inspired by the eponymous highway that runs through the city of Guelph, and signifies the team's ties to the University of Guelph, as well as the city of Guelph.\\

\begin{figure}[H]
	\centering	
	\includegraphics[height=2in]{img/hanlon-logo.png}
	%\caption{The team logo}
	\label{fig:kitten}
\end{figure}

\subsection{Team Members}
Hanlon is comprised of the following students:\\
1. \textbf{\hspace*{5pt} David DiMaria} - Project Manager\\
2. \textbf{\hspace*{5pt}Braydon Johnson} - Software Developer, User Interface Designer\\
3. \textbf{\hspace*{5pt}Joshua Lemieux} - Project Manager, User Interface Designer\\
4. \textbf{\hspace*{5pt}Neivin Mathew} - Software Developer, User Interface Designer\\
5. \textbf{\hspace*{5pt}Like Zheng} - Software Developer

\subsection{Team Roles}
\subsubsection*{Project Manager}
The Project Manager predicts potential problems that may arise during development, and plans tasks to ensure that the project is completed successfully and on time. This role involves the scheduling and unblocking of tasks. It may also involve some programming.

\subsubsection*{Software Developer}
The Developer is involved in all aspects of the software development process including research, design, coding, documentation and testing.

\subsubsection*{User Interface Designer}
The User Interface (UI) Designer role is to plan out and develop any user facing component of the system which includes the specific layout of screens, and improving the interaction between the customer and the product.

\subsection{Team Organization}
Hanlon will follow a static team structure. Each member will maintain their respective roles for the entire duration of the project. \par

Hanlon will use a democratic majority voting system for any decisions that need to be taken within the team. Each present member will be involved in voting, and possesses one vote per motion. A motion is passed when a simple majority is achieved. \par

In the event of a team member being unavailable, and a majority cannot be established, a motion can only be passed through unanimous consent.

\clearpage
\section{Requirements}
\subsection{Definitions}
The terms used in the requirements document are defined as follows:\\
1. \hspace*{5pt} \textbf{ARET -- } The Agricultural Research and Extension Trust of Malawi. \\
2. \hspace*{5pt} \textbf{SMS -- } Short Message Service. A service on cell phones that allows the exchange of short text messages.\\
3. \hspace*{5pt} \textbf{API -- } Application Programming Interface. A set of tools that allows applications to interact with each other.\\
4. \hspace*{5pt} \textbf{SQL -- } Structured Query Language. A language that allows the definition and manipulation of data.\\
5. \hspace*{5pt} \textbf{GUI -- } Graphical User Interface. A visual framework that enables easy interaction with an application.

\subsection{Multiple Dependencies}
The following requirements have more than one dependency: \\
1. \textbf{Requirement \#027 -} Depends on requirement \#016 and requirement \#013. \\
2. \textbf{Requirement \#029 -} Depends on requirement \#017 and requirement \#013. \\
3. \textbf{Requirement \#030 -} Depends on requirement \#016 and requirement \#014. \\
4. \textbf{Requirement \#035 -} Depends on requirement \#016 and requirement \#015.

\subsection{Requirements Table}
The table of requirements can be found in the file named "\texttt{CIS3750\char`_A2\char`_Hanlon.csv}" file attached to the report.

\clearpage
\section{Use Cases}
\subsection{•}

\clearpage
\section{Time Estimation}
Assuming the team begins coding the project on October 17, 2016 and finishes the project before December 5, 2016, the team has a total of 49 days to complete the work. \par
Excluding weekends, the team has 35 working days (WDs). Working at a velocity of 80\%, there are 28 productive working days (PWDs) in this period. \par
The project would take 166.5 productive working days for one person to complete, or 209 (208.125) working days. For a team of five, the project would take 36 PWDs, or 45 working days to complete. \par
The project would need 10 more working days to complete than the allotted time period of 35 working days. The project would have the majority of high priority features (must, should), while a few low priority features (could) would not be completed. Approximately 8\% of the total requirements outlined for the project would not be completed (12 out of 144).

\subsection{Project Timeline}
The project timeline can be found in the file named "\texttt{CIS3750\char`_A2\char`_TimeLine\char`_Hanlon.csv}" file attached to the report.

\clearpage
\section{Individual Contributions}
\subsection{Josh Lemieux}
\textbf{For the Price of a Latte}\par


\clearpage
\subsection{David DiMaria}
\textbf{Knowing is a Right}

\clearpage
\subsection{Like Zheng}
\textbf{What is the Real Challenge?}\par


\clearpage
\subsection{Neivin Mathew}
\textbf{The Advancement of Prosthetics}\par

\clearpage
\subsection{Braydon Johnson}
\textbf{The Problem with Today's Education System}\par


\clearpage
\section{References}
\begin{flushleft}
\begin{itemize}[leftmargin=12pt]

\item World Food Programme (2016). \emph{Malawi.}
 Retrieved from \texttt{https://www.wfp.org/countries/malawi}

\item Agricultural Research and Extension Trust of Malawi (2016). \emph{About ARET Malawi.}
Retrieved from \texttt{http://www.aret.org.mw/index.php/about-us/profile}

\item Agricultural Research and Extension Trust of Malawi (2016, January). \emph{ARET Strategic Plan 2016-2021.}


\end{itemize}
\end{flushleft}   



\end{document}