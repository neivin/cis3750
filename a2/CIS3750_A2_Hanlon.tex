\documentclass[12pt,letterpaper]{article}
\usepackage[utf8]{inputenc}
\usepackage{kpfonts}
\usepackage[T1]{fontenc}

% custom titles
\usepackage{titlesec}

% fix broken title numbering with 2016 titlesec update
\usepackage{etoolbox}

\makeatletter
\patchcmd{\ttlh@hang}{\parindent\z@}{\parindent\z@\leavevmode}{}{}
\patchcmd{\ttlh@hang}{\noindent}{}{}{}
\makeatother
%%%

\titlespacing*\section{0pt}{12pt plus 4pt minus 2pt}{0pt plus 2pt minus 2pt}
\titlespacing*\subsection{0pt}{12pt plus 4pt minus 2pt}{0pt plus 2pt minus 2pt}
\titlespacing*\subsubsection{0pt}{12pt plus 4pt minus 2pt}{0pt plus 2pt minus 2pt}

% Package for double spacing
\usepackage{setspace}
\usepackage{ragged2e}

% Set 1.0 inch margins
\usepackage[margin=1.0in, headheight=15pt]{geometry}
\usepackage{enumitem}

% Use images and graphics
\usepackage{graphicx}
\usepackage{float}

% Use nicer headers
\usepackage{fancyhdr}
\pagestyle{fancy}
\renewcommand{\headrulewidth}{0pt}
\rhead{CIS3750 - Assignment 1}

% sections should be indexed with alphabets
\renewcommand{\thesection}{\Alph{section}}

%double spaced lines in the whole document
\doublespacing

\title{Assignment 1}

\begin{document}
\begin{titlepage}
    \centering
    \vspace*{\baselineskip}
    \rule{\textwidth}{1.6pt}\vspace*{-\baselineskip}\vspace*{2pt}
    \rule{\textwidth}{0.4pt}\\[1.5\baselineskip]
    {\LARGE \textsc{An Initial Outline of a Software System to Improve Food Security in Malawi}}\\[\baselineskip]
	\rule{\textwidth}{0.4pt}\vspace*{-\baselineskip}\vspace{4pt}    
    \rule{\textwidth}{2pt}\\[2\baselineskip]
   
    \vspace*{5\baselineskip}
    \textsc{BY}\\[0.25\baselineskip]
    {\LARGE HANLON} \\
    
    \vspace*{\baselineskip}
    % List of authors in alphabetical order (by last name)
    {\textsc{David DiMaria \\ Braydon Johnson \\ Joshua Lemieux \\ Neivin Mathew \\ Like Zheng} \par}
    \vfill
    {\scshape October 14, 2016} \\
  \end{titlepage}
  
  
% Table of Contents (no page numbers on contents)
\pagenumbering{roman} %roman numerals for ToC
\tableofcontents
\lhead{} % remove default header from Contents page
\clearpage
\pagenumbering{arabic} %pagenumbering in arabic numbers
    
\section{Client Details}
Malawi is a country located in the warm heart of Africa, with a population of 16.4 million. According to the Malawi Vulnerability Assessment Committee, an estimated 2.83 million people will experience acute food insecurity during the 2015/16 lean season. (World Food Programme, 2016) The economy of the country is based on agriculture. The majority of the people in Malawi are farmers who cultivate tobacco for living. Due to the impact of climate change, decline in global tobacco consumption and the scarcity of food in Malawi, the farmers need to transition to growing different crops. However, since Malawian farmers have been growing tobacco for many generations, the current farmers of the country lack the tools and knowledge required in order to start growing other crops. 

The client for the project is the Agricultural Research and Extension Trust (ARET) of Malawi. They are the premier research institution of Malawi, and are responsible for conducting research and providing technical/extension services on tobacco. The trust was established on September 1, 1995 to foster development and information dissemination for Malawi's tobacco industry. It combined the services of two institutions, the Tobacco Research Institute of Malawi (TRIM) and the Estate Extension Service Trust (EEST), who separately provided research and extension services respectively. (ARET, 2016)

ARET states their new vision is "to be a leading regional centre of excellence in agricultural research and technology dissemination which promotes diversification in the agricultural sector." (ARET, 2016). To accomplish this, they require a software system to collect data on the farmers in Malawi, and distribute research conducted within the organization to the farmers of the country. This project aims to fulfil this need.

\clearpage
\section{Team Details}
\subsection{Team Name}
The team name for the project is "Hanlon."\par
The name is inspired by the eponymous highway that runs through the city of Guelph, and signifies the team's ties to the University of Guelph, as well as the city of Guelph.\\

\begin{figure}[H]
	\centering	
	\includegraphics[height=2in]{img/hanlon-logo.png}
	%\caption{The team logo}
	\label{fig:kitten}
\end{figure}

\subsection{Team Members}
Hanlon is comprised of the following students:\\
1. \textbf{\hspace*{5pt} David DiMaria} - Project Manager\\
2. \textbf{\hspace*{5pt}Braydon Johnson} - Software Developer, User Interface Designer\\
3. \textbf{\hspace*{5pt}Joshua Lemieux} - Project Manager, User Interface Designer\\
4. \textbf{\hspace*{5pt}Neivin Mathew} - Software Developer, User Interface Designer\\
5. \textbf{\hspace*{5pt}Like Zheng} - Software Developer

\subsection{Team Roles}
\subsubsection*{Project Manager}
The Project Manager predicts potential problems that may arise during development, and plans tasks to ensure that the project is completed successfully and on time. This role involves the scheduling and unblocking of tasks. It may also involve some programming.

\subsubsection*{Software Developer}
The Developer is involved in all aspects of the software development process including research, design, coding, documentation and testing.

\subsubsection*{User Interface Designer}
The User Interface (UI) Designer role is to plan out and develop any user facing component of the system which includes the specific layout of screens, and improving the interaction between the customer and the product.

\subsection{Team Organization}
Hanlon will follow a static team structure. Each member will maintain their respective roles for the entire duration of the project. \par

Hanlon will use a democratic majority voting system for any decisions that need to be taken within the team. Each present member will be involved in voting, and possesses one vote per motion. A motion is passed when a simple majority is achieved. \par

In the event of a team member being unavailable, and a majority cannot be established, a motion can only be passed through unanimous consent.

\clearpage
\section{Requirements}
\subsection{Definitions}
The terms used in the requirements document are defined as follows:\\
1. \hspace*{5pt} \textbf{ARET -- } The Agricultural Research and Extension Trust of Malawi. \\
2. \hspace*{5pt} \textbf{SMS -- } Short Message Service. A service on cell phones that allows the exchange of short text messages.\\
3. \hspace*{5pt} \textbf{API -- } Application Programming Interface. A set of tools that allows applications to interact with each other.\\
4. \hspace*{5pt} \textbf{SQL -- } Structured Query Language. A language that allows the definition and manipulation of data.\\
5. \hspace*{5pt} \textbf{GUI -- } Graphical User Interface. A visual framework that enables easy interaction with an application.

\subsection{Multiple Dependencies}
The following requirements have more than one dependency: \\
1. \textbf{Requirement \#027 -} Depends on requirement \#016 and requirement \#013. \\
2. \textbf{Requirement \#029 -} Depends on requirement \#017 and requirement \#013. \\
3. \textbf{Requirement \#030 -} Depends on requirement \#016 and requirement \#014. \\
4. \textbf{Requirement \#035 -} Depends on requirement \#016 and requirement \#015.

\subsection{Requirements Table}
The table of requirements can be found in the file named "\texttt{CIS3750\char`_A2\char`_Hanlon.csv}" file attached to the report.

\clearpage
\section{Use Cases}

\subsection{Upload Research Document for Extension}
\subsubsection*{Primary Actor:} ARET Researcher
\subsubsection*{Stakeholders and Interests:}
1. \emph{ARET Researcher --} wants to upload original research documents to be extended, wants the process to be clear and simple.\\[10pt]
2. \emph{ARET Extension Group --} wants to download original research documents, , wants the process to be clear and simple.

\subsubsection*{Preconditions:}
The researcher is identified and authenticated by the system and has a file ready to be uploaded.

\subsubsection*{Success Guarantee (Postconditions):}
The Researcher successfully uploads a research file to the web portal for the Extension Group to access. List of research files to be extended is updated.

\subsubsection*{Main Success Scenario:}
1. The user indicates to the system that they want to upload a new file for extension.\\
2. The system redirects the user to a page where they are prompted to choose a file from their device or exit the use case [Use case ends].\\
3. The user selects a new file from their device to be uploaded and confirms that the file is correct. \emph{[Alt1: Incorrect file selected]}\\
4. The system validates the selected file on certain criteria such as file format, file size, etc. \emph{[Alt2: file is not correct format][Alt3: file is over size limit]}\\
5. The system uploads the file selected by the user to the system database. 


\subsubsection*{Alternative Flows:}
\emph{Alt1:Incorrect file selected}\\
1. Flow resumes at Main Success Scenario Step 2. \\[10pt]
\emph{Alt2: File is not in correct format}\\
1. The system informs the user that the selected file is of an invalid file format. \\
2. Flow resumes at Main Success Scenario Step 2. \\[10pt]
\emph{Alt3: File exceeds size restriction}\\
1. The system informs the user that the selected file exceeds the size restriction on uploaded files.\\
2. Flow resumes at Main Success Scenario Step 2.


\subsubsection*{Exceptions:}
If at any time the system is unable to authenticate the user, or unable to upload the research document to the database due to failure of the system API or the user’s internet connection, the system informs the user of the problem and the use case ends.


\clearpage
\subsection{View Original Research Document for Extension}
\subsubsection*{Primary Actor:} ARET Extension Group
\subsubsection*{Stakeholders and Interests:}
1. \emph{ARET Extension Group --} wants to download the research data to convert into extension materials, wants accurate research data to convert, wants the process to be clear and simple.\\[10pt]
2. \emph{ARET Researcher --} wants the extension materials generated from the research to accurately convey the research data, wants to know how their research is being used. \\[10pt]
3. \emph{Farmer --} wants the research data to be converted into extension materials, wants to apply the information from the research to their own farm.

\subsubsection*{Preconditions:}
Extension Group member is identified and authenticated by the system, research document exists in the system database.

\subsubsection*{Success Guarantee (Postconditions):}
The Extension Group member successfully downloads a research file from the web portal to their device. The research file data is updated to indicate that it is currently being worked on by the Extension Group.

\subsubsection*{Main Success Scenario:}
1. The user requests a list of research documents that are currently available for conversion to extension materials. \emph{[Alt1: No research documents available]}\\
2. The system retrieves the list of research documents available and displays the list to the user.\\
3. The system provides the user with the choice to select a document or to exit the use case. \emph{[Use case ends]}\\
4. The user selects a research document they want to view.\\
5. The system redirects the user to a page outlining the details of the document (File size, Author, Date of Last Update etc.) \\
6. The user indicates to the system that they would like to either view the selected document, download it to their device \emph{[Alt2: Download research document]}, or select a different research document from the list \emph{[Alt3: Select different research document]}.\\ 
7. The system fetches the document from the ARET database and displays it to the user through their browser.

\subsubsection*{Alternative Flows:}
\emph{Alt1: No research documents available}\\
1. The system informs the user that no research documents are currently available for conversion to extension materials. Use case ends.\\[10pt]
\emph{Alt2: Download research document}\\
1. The system fetches the desired document from the ARET database and downloads the file to the default download directory on the mobile device. \\
2. The system informs the user that the file has been successfully downloaded to the device. \\[10pt]
\emph{Alt3: Select different research document}\\
1. Flow resumes at Main Success Scenario Step 2.

\subsubsection*{Exceptions:}
If at any time the system is unable to authenticate users or retrieve research documents from the system database due to failure of the system API or the user’s internet connection, the system informs the user of the problem and the use case ends.

\clearpage
\subsection{Create a Farmer Account}
\subsubsection*{Primary Actor:} Farmer
\subsubsection*{Stakeholders and Interests:}
1. \emph{Farmer --} wants to create an account, wants to access farming information and research data, wants the process to be clear and simple. \\[10pt]
2. \emph{ARET --} wants the system to correctly record farmer account creation for accurate data collection.
\subsubsection*{Preconditions:}
The mobile application must be built and the API for user creation must be stable.

\subsubsection*{Success Guarantee (Postconditions):}
The Farmer is aware that their account has been created. The user database is updated. Farmer is able to login and access the mobile application content.

\subsubsection*{Main Success Scenario:}
1. The user indicates to the System that they would like to create an account.\\
2. The system verifies that the mobile device is in Malawi to prevent abuse of the System. \emph{[Alt1: Mobile device is not in Malawi]}\\
3. The system redirects the user to an account creation page where they enter in the required information that they are prompted for, and indicate to the System when they are finished creating an account . The user can choose to cancel the creation process and exit the use case. \emph{[Use Case Ends]}\\
4. The system validates the new account information and then returns a message to the user that their account has been created. \emph{[Alt2: Account name fails validation][Alt3: Account password fails validation]}\\
5. The user is redirected to the main application dashboard by the system.

\subsubsection*{Alternative Flows:}
\emph{Alt1: Mobile device is not in Malawi}\\
1.  The system informs the user that their location is invalid and are unable to create an account. \\
2. The system reminds the user to turn on their location services. Use case ends.\\[10pt]
\emph{Alt2: Account name fails validation}\\
1. The system informs the user that their account name either already exists or contains invalid characters. \\
2. Flow resumes at Step 3. \\[10pt]
\emph{Alt3:  Account password fails validation}\\
1. The system informs the user that their account password contains invalid characters or does not meet the criteria for a secure password.
2. Flow resumes at Step 3.

\subsubsection*{Exceptions:}
If at any time the system is unable to call the API, or the user’s internet connection fails, the system informs the user of the problem, and the use case ends.

\clearpage
\subsection{Login to a Farmer Account}
\subsubsection*{Primary Actor:} Farmer
\subsubsection*{Stakeholders and Interests:}
1. \emph{Farmer --} wants to login to their account, wants to access farming information and research data, wants the process to be clear and simple. \\[10pt]
2. \emph{ARET Researcher --} wants accurate lists of farmers registered on the mobile application, wants to filter the list of farmers based on different criteria, wants to be able to send notifications to a list of farmers filtered based on some criteria, wants to know the usage information for the mobile application.
\subsubsection*{Preconditions:}
The Farmer has the mobile application installed, and has created an account. The system API is stable.

\subsubsection*{Success Guarantee (Postconditions):}
The Farmer is logged into their account and able to access the features that being logged in enables.

\subsubsection*{Main Success Scenario:}
1. The user inputs their account information into the system or chooses to exit the use case \emph{[Use case ends]}.\\
2. The user indicates to the System that they would like to login or recover a lost account. \emph{[Alt1: Recover account details]}\\
3. The system verifies that the user's login information is valid. \emph{[Alt2: account name invalid][Alt3: Password is invalid]}\\
4. The system redirects the user the main landing page of the mobile application.


\subsubsection*{Alternative Flows:}
\emph{Alt1:  Recover account details}\\
1. The system redirects the user to a password reset landing page.\\
2. The user inputs their account name and indicates to the system that they would like to reset their account password. \\
3. The system verifies that the account name is valid, and generates a temporary random password and sends an email to the user notifying them of the change.\\
4. The system informs the user that their account password has been reset and advises them to check their email for details.\\[10pt]
\emph{Alt2: Account name is unrecognized}\\
1. The system informs the user that the account name is not registered in the system.\\
2. The system refreshes the page and prompts for a registered account name.\\
3. Flow resumes at Step 1.\\[10pt]
\emph{Alt3: Password is unrecognized}\\
1. The system informs the user that the password does not match the account name.\\
2. Flow resumes at Step 1.

\subsubsection*{Exceptions:}
If at any time the system is unable to retrieve the user’s account details for validation due to failure of the API or internet connection, the system informs the user of the problem and the use case ends.

\clearpage
\subsection{View Extended Research Document}
\subsubsection*{Primary Actor:} Farmer
\subsubsection*{Stakeholders and Interests:}
1. \emph{Farmer:} wants to view and download research data, wants to be able to apply information from the research to their farm, wants the process to be clear and simple. \\[10pt]
2. \emph{ARET Researcher:} wants accurate lists of farmers registered on the mobile application, wants to filter the list of farmers based on different criteria, wants to be able to send notifications to a list of farmers filtered based on some criteria, wants to know the usage information for specific research data.\\[10pt]
3. \emph{ARET Extension Group:} wants accurate lists of farmers registered on the mobile application, wants to know download and usage metrics for extension materials.

\subsubsection*{Preconditions:}
Farmer has created an account, has been identified and authenticated by the system. Extension group has posted reasearch files.

\subsubsection*{Success Guarantee (Postconditions):}
Farmer has viewed desired research and has optionally downloaded the research information onto their device for offline viewing. 

\subsubsection*{Main Success Scenario:}
1. The user informs the system that they want to view a catalog of all research materials or view already downloaded research materials. \emph{[Alt1: View previously downloaded research materials]}\\
2. The user searches for the desired crop/farming technique for which they want research information. \emph{[Alt2: No search results found]}\\
3. The user selects the specific research document from the list of search results.
4. The system redirects the user to a page outlining the details of the document (File size, Author, Date of Last Update etc.)
5. The user indicates to the system that they would like to either view the selected document or download it for offline viewing. \emph{[Alt3: Download research document]}\\
6. The system fetches the document from the ARET database and displays it to the user through the default document viewer on the mobile device.

\subsubsection*{Alternative Flows:}
\emph{Alt : View previously downloaded research materials}\\
1. The system redirects the user to a page with a list of all downloaded research files.\\
2. The user selects a research document they want to view\\
3. The system displays the research file to the user using their default document viewer.\\[10pt]
\emph{Alt : No search results found}\\
1. The system informs the user that no research documents were found that matched their search criteria. Use case ends. \\[10pt]
\emph{Alt : Download research document}\\
1. The system fetches the desired document from the ARET database and downloads the file to the default download directory on the mobile device.\\
2. The system informs the user that the file has been successfully downloaded to the device.

\subsubsection*{Exceptions:}
If at any time the system is unable to authenticate users or retrieve research documents from the system database due to failure of the system API or the user’s internet connection, the system informs the user of the problem and the use case ends.

\clearpage
\section{Time Estimation}
Assuming the team begins coding the project on October 17, 2016 and finishes the project before December 5, 2016, the team has a total of 49 days to complete the work. \par
Excluding weekends, the team has 35 working days (WDs). Working at a velocity of 80\%, there are 28 productive working days (PWDs) in this period. \par
The project would take 166.5 productive working days for one person to complete, or 209 (208.125) working days. For a team of five, the project would take 36 PWDs, or 45 working days to complete. \par
The project would need 10 more working days to complete than the allotted time period of 35 working days. The project would have the majority of high priority features (must, should), while a few low priority features (could) would not be completed. Approximately 8\% of the total requirements outlined for the project would not be completed (12 out of 144).

\subsection{Project Timeline}
The project timeline can be found in the file named "\texttt{CIS3750\char`_A2\char`_TimeLine\char`_Hanlon.csv}" file attached to the report.

\clearpage
\section{Individual Contributions}
\subsection{Josh Lemieux}
\textbf{For the Price of a Latte}\par


\clearpage
\subsection{David DiMaria}
\textbf{Knowing is a Right}

\clearpage
\subsection{Like Zheng}
\textbf{What is the Real Challenge?}\par


\clearpage
\subsection{Neivin Mathew}
\textbf{The Advancement of Prosthetics}\par

\clearpage
\subsection{Braydon Johnson}
\textbf{The Problem with Today's Education System}\par


\clearpage
\section{References}
\begin{flushleft}
\begin{itemize}[leftmargin=12pt]

\item World Food Programme (2016). \emph{Malawi.}
 Retrieved from \texttt{https://www.wfp.org/countries/malawi}

\item Agricultural Research and Extension Trust of Malawi (2016). \emph{About ARET Malawi.}
Retrieved from \texttt{http://www.aret.org.mw/index.php/about-us/profile}

\item Agricultural Research and Extension Trust of Malawi (2016, January). \emph{ARET Strategic Plan 2016-2021.}


\end{itemize}
\end{flushleft}   



\end{document}