\documentclass[12pt,letterpaper]{article}
\usepackage[utf8]{inputenc}
\usepackage{kpfonts}
\usepackage[T1]{fontenc}

\usepackage{titlesec}

\titlespacing*\section{0pt}{12pt plus 4pt minus 2pt}{0pt plus 2pt minus 2pt}
\titlespacing*\subsection{0pt}{12pt plus 4pt minus 2pt}{0pt plus 2pt minus 2pt}
\titlespacing*\subsubsection{0pt}{12pt plus 4pt minus 2pt}{0pt plus 2pt minus 2pt}

% Package for double spacing
\usepackage{setspace}

% Set 1.0 inch margins
\usepackage[margin=1.0in, headheight=15pt]{geometry}



% Use images and graphics
\usepackage{graphicx}
\usepackage{float}

% no item indent
\usepackage{enumitem}

% Use nicer headers
\usepackage{fancyhdr}
\pagestyle{fancy}
\renewcommand{\headrulewidth}{0pt}
\rhead{CIS3750 - Assignment 1}

\doublespacing

\title{Assignment 1}

\begin{document}
\begin{titlepage}
    \centering
    \vspace*{\baselineskip}
    \rule{\textwidth}{1.6pt}\vspace*{-\baselineskip}\vspace*{2pt}
    \rule{\textwidth}{0.4pt}\\[1.5\baselineskip]
    {\LARGE \textsc{ARET Food Security System}}\\[\baselineskip]
	\rule{\textwidth}{0.4pt}\vspace*{-\baselineskip}\vspace{4pt}    
    \rule{\textwidth}{2pt}\\[2\baselineskip]
   
    \vspace*{5\baselineskip}
    \textsc{by}\\[0.25\baselineskip]
    {\LARGE HANLON} \\
    
    \vspace*{\baselineskip}
    % List of authors in alphabetical order (by last name)
    {\textsc{David DiMaria \\ Braydon Johnson \\ Joshua Lemieux \\ Neivin Mathew \\ Like Zheng} \par}
    \vfill
    {\scshape Fall 2016} \\   
  \end{titlepage}
  
  
% Table of Contents (no page numbers on contents)
\pagenumbering{roman} %roman numerals for ToC
\tableofcontents
\lhead{} % remove default header from Contents page
\clearpage
\pagenumbering{arabic} %pagenumbering in arabic numbers
    
\section{Client Details}
Malawi is a country located in the warm heart of Africa, with a population of 16.4 million. Last year, there were 1 million people in Malawi facing a problem known as food insecurity. This year, that number will rise to 6.4 million. The economy of the country is based on agriculture. Most of the people in Malawi are farmers who grow tobacco for living. Due to the impact of climate change and the growing lack of food, the farmers need to transition to growing different crops. The problem is that many of the Malawians have never grown anything besides tobacco, and do not contain the knowledge to start growing other crops. This is the issue that ARET are trying to solve. \par

ARET (Agricultural Research and Extension Trust) is a research organization in Malawi. They are Malawi’s premier research institution, which are  responsible for conducting research and providing technical/extension services on tobacco. The Trust was established on September 1st, 1995 to foster development and information dissemination for Malawi’s tobacco industry. It amalgamated the services of two institutions, Tobacco Research Institute of Malawi (TRIM) and the Estate Extension Service Trust (EEST), who separately provided research and extension services respectively (reference 1). \par

ARET states their vision as “To be a leading regional centre of excellence in agricultural research and technology dissemination which promotes diversification in the agricultural sector.” (ARET Strategic Plan 2016-2021). 

\section{Team Details}
\begin{figure}[H]
	\centering	
	\includegraphics[height=2in]{img/hanlon-logo.png}
	%\caption{The Company Logo}
	\label{fig:kitten}
\end{figure}

\subsection{Team Members}
Hanlon is comprised of the following students:\\
1. \textbf{\hspace*{5pt} David DiMaria} - Project Manager\\
2. \textbf{\hspace*{5pt}Braydon Johnson} - Software Developer, User Interface Designer\\
3. \textbf{\hspace*{5pt}Joshua Lemieux} - Project Manager, User Interface Designer\\
4. \textbf{\hspace*{5pt}Neivin Mathew} - Software Developer, User Interface Designer\\
5. \textbf{\hspace*{5pt}Like Zheng} - Software Developer

\subsection{Team Roles}
\subsubsection*{Project Manager}
The Project Manager predicts potential problems that may arise during development, and plans tasks to ensure that the project is completed successfully and on time. This role involves the scheduling and unblocking of tasks. It may also involve some programming.

\subsubsection*{Software Developer}
The Developer is involved in all aspects of the software development process including research, design, coding, documentation and testing.

\subsubsection*{User Interface Designer}
The User Interface (UI) Designer role is to plan out and develop any user facing component of the system which includes the specific layout of screens, and improving the interaction between the customer and the product.

\subsection{Team Organization}
Hanlon will follow a static team structure. Each member will maintain their respective roles for the entire duration of the project. \par

Hanlon will use a democratic majority voting system for any decisions that need to be taken within the team. Each present member will be involved in voting, and possesses one vote per motion. A motion is passed when a simple majority is achieved. \par

In the event of a team member being unavailable, and a majority cannot be established, a motion can only be passed through unanimous consent.

\clearpage
\section{Project Goals and Users}
\subsection{Project Goals}

The goal of the project is to design and develop a system that allows the Agricultural Research and Extension Trust (ARET) of Malawi to provide agricultural information to the farmers of Malawi, and relay the data collected from the system back to ARET and its partners.

\subsection{Users}
1.\hspace*{5pt} \textbf{System Administrator -- } Maintains database and API.\\
2.\hspace*{5pt} \textbf{Researcher -- } Creates research materials and reviews extension materials. \\
3.\hspace*{5pt} \textbf{Extension Service --} Creates resources for farmers from research materials. \\
4.\hspace*{5pt} \textbf{Extension Officer -- } Acts as the liaison between the Researcher and Extension Agent \\
5.\hspace*{5pt} \textbf{Extension Agent -- } Acts as the liaison between the Extension Officer and the Farmer. \\
6.\hspace*{5pt} \textbf{Farmer -- } Views information resources , supplies farm data, and responds to surveys. \\
7.\hspace*{5pt} \textbf{Partner -- } Collaborates with ARET, and may access the database and web portal \\
8.\hspace*{5pt} \textbf{Public -- } Uses the system without a user account. Includes any individual that does not fit into the other user groups. \\
9.\hspace*{5pt} \textbf{System -- } Consists of the entire platform (web, mobile), database, and API.

\subsection{Project Organization}
Version control of the codebase will be done using Git. Trello will be used to keep track of sprints and other project management tasks. Communication within the team and between teams will be accomplished through Slack.\par
The database will be built with SQL, using MySQL or PostgreSQL. The API would be developed in Python on the Django REST Framework. The web portal and mobile application will be built concurrently using HTML5, CSS3, and JavaScript, while relying on the Adobe PhoneGap framework to create a hybrid platform.


\subsubsection*{System Administrator}
The System Administrator is a member of ARET responsible for managing user accounts, maintaining the database and monitoring the usage of the Application Programming Interface (API). They are in close communication with the development team, and report structural changes the database and API.

\subsubsection*{Researcher}
A Researcher is a member of ARET responsible for creating and managing research projects within the system. Additionally, they will review the materials created by the Extension Service to ensure that they accurately convey research findings.

\subsubsection*{Extension Service}
The Extension Service is the part of ARET responsible for using the results from the research projects to create resources that can be easily accessed by the farmers of Malawi.

\subsubsection*{Extension Officer}
An Extension Officer is a member of ARET responsible for communicating with the Extension Service through the system and distributing relevant extension materials to their respective Agents. They are also responsible for collecting and validating data from the Farmers and Extension Agents.

\subsubsection*{Extension Agent}
An Extension Agent is a member of ARET responsible for communicating with their Extension Officer within the system and distributing relevant extension materials to the Farmers within their assigned region. They are also responsible for collecting data from the Farmer.

\subsubsection*{Farmer}
A Farmer is a citizen of Malawi involved in agriculture. They are the primary users of the mobile application, which will allow them to track their farm statistics, receive notifications from ARET, respond to ARET sanctioned surveys, and report incidents to ARET.

\subsubsection*{Partner}
A Partner is any organization or entity working closely with ARET. This includes, but is not limited to, agencies such as the Farm Radio Trust, mHub, the African Forum for Agricultural Advisory Services (AFAAS), Left, the University of Guelph, etc.

\subsubsection*{Public}
The Public is anyone who is using the web application but is not one of the defined users. This includes potential donors, non Malawians, other Malawian citizens, etc.

\subsubsection*{System}
The System refers to the entire software platform including the web portal, mobile application, data collection platform, the farmer crop management database and API.


\clearpage
\section{Individual Contributions}
\clearpage
\subsection{David DiMaria}
\textbf{Collaboration is the New Competitive Advantage}

After reading the articles provided, there was a lot to think about in regards to how I am going to use my degree after I graduate. Obviously I’d like to be employed by a software company, and make good money. However, I hadn’t given much thought to where I would want to work within computer science, and whether or not it was important that I’m passionate about what I’m working on professionally. If I could use my skills towards some world challenge, it would be the right to knowledge, and specifically the right to free internet. To elaborate on that, I grew up in the age where by the time I was able to type, search engines were extremely popular and any information that I wanted was right at my fingertips. I could know anything at any moment, uncensored, as long as I had a stable internet connection. I could learn about anything I want, immerse myself in online communities, share information with my peers and find like minded individuals from all around the world.\par

This is a challenge since a large portion of the world doesn’t have internet access, and many that do have portions censored based on which country they're from. This is something that people should care about because if impoverished people can learn anything at any moment, they could be saved from governments trying to suppress them physically and intellectually. The reason why governments will try and censor things is because of the amount of political unrest it would cause given there people had all the information. The power that information could have on a population is unfathomable, because they could learn the truth about how their government is perceived in the eyes of the rest of the world.\par

If I could choose a group of people to help me achieve this goal, it would be Google, Facebook, and telecommunications companies. The power of internet infrastructure at companies like Google and Facebook is large enough that they could develop a worldwide satellite network with telecommunications companies that delivers internet to everyone.


\end{document}