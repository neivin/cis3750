\documentclass[12pt,letterpaper]{article}
\usepackage[utf8]{inputenc}
\usepackage{kpfonts}
\usepackage[T1]{fontenc}

% custom titles
\usepackage{titlesec}

% fix broken title numbering with 2016 titlesec update
\usepackage{etoolbox}

\makeatletter
\patchcmd{\ttlh@hang}{\parindent\z@}{\parindent\z@\leavevmode}{}{}
\patchcmd{\ttlh@hang}{\noindent}{}{}{}
\makeatother
%%%

\titlespacing*\section{0pt}{12pt plus 4pt minus 2pt}{0pt plus 2pt minus 2pt}
\titlespacing*\subsection{0pt}{12pt plus 4pt minus 2pt}{0pt plus 2pt minus 2pt}
\titlespacing*\subsubsection{0pt}{12pt plus 4pt minus 2pt}{0pt plus 2pt minus 2pt}

% Package for double spacing
\usepackage{setspace}
\usepackage{ragged2e}

% Set 1.0 inch margins
\usepackage[margin=1.0in, headheight=15pt]{geometry}
\usepackage{enumitem}

% Use images and graphics
\usepackage{graphicx}
\usepackage{float}

\usepackage{pgf-umlcd}
\usepackage{pgf-umlsd}

% Use nicer headers
\usepackage{fancyhdr}
\pagestyle{fancy}
\renewcommand{\headrulewidth}{0pt}
\rhead{CIS3750 Final Report}

% sections should be indexed with alphabets
\renewcommand{\thesection}{\Alph{section}}

%double spaced lines in the whole document
\doublespacing

\title{Final Report}

\begin{document}
\begin{titlepage}
    \centering
    \vspace*{\baselineskip}
    \rule{\textwidth}{1.6pt}\vspace*{-\baselineskip}\vspace*{2pt}
    \rule{\textwidth}{0.4pt}\\[1.5\baselineskip]
    {\LARGE \textsc{Reflections on a Software Prototype}}\\[\baselineskip]
	\rule{\textwidth}{0.4pt}\vspace*{-\baselineskip}\vspace{4pt}    
    \rule{\textwidth}{2pt}\\[2\baselineskip]
   
    \vspace*{5\baselineskip}
    \textsc{BY}\\[0.25\baselineskip]
    {\LARGE HANLON} \\
    
    \vspace*{\baselineskip}
    % List of authors in alphabetical order (by last name)
    {\textsc{David DiMaria \\ Braydon Johnson \\ Joshua Lemieux \\ Neivin Mathew \\ Like Zheng} \par}
    \vfill
    {\scshape December 5, 2016} \\
  \end{titlepage}
  
  
% Table of Contents (no page numbers on contents)
\pagenumbering{roman} %roman numerals for ToC
\tableofcontents
\lhead{} % remove default header from Contents page
\clearpage
\pagenumbering{arabic} %pagenumbering in arabic numbers
    
\clearpage
\section{Video}
The video produced by our team may be found on YouTube at the link provided in the file \texttt{CIS3750\_Video\_Hanlon.txt} .

\clearpage
\section{Class Diagram}
The class diagram can be found at the end of this document.\\
It is also included as a separate file: \texttt{CIS3750\_ClassDiagram\_Hanlon.pdf}


\section{Sequence Diagrams}

\subsection{Login to a Farmer Account}
\begin{center}
\begin{sequencediagram}
\def\unitfactor{0.8}
	\newthread{usr}{User}
	\newinst[3]{app}{Application}
	\newinst[5]{svr}{Server}

	\begin{call}{usr}{Input Credentials}{app}{Display Main Page}
		\begin{call}{app}{sendCredentials(email, password)}{svr}{loginStatus}
		\end{call}	
	\end{call}	
\end{sequencediagram}
\end{center}


\subsection{Add a Crop to Farmer Records}
\begin{center}
\begin{sequencediagram}
\def\unitfactor{0.8}
	\newthread{usr}{User}
	\newinst[3]{app}{Application}
	\newinst[5]{svr}{Server}

	\begin{call}{usr}{Click Add Button}{app}{Update Search Page}
		\begin{call}{app}{getCroplist()}{svr}{cropList}
		\end{call}	
	\end{call}
	
	\begin{call}{usr}{Search for Crop}{app}{Display Results}
	\end{call}	
	
	
	\begin{call}{usr}{Choose Crop}{app}{Update Crops Page}
		\begin{call}{app}{addCropToProfile(cropName)}{svr}{status}
		\end{call}	
	\end{call}
\end{sequencediagram}
\end{center}


\subsection{Download Research Document}
\begin{center}
\begin{sequencediagram}
\def\unitfactor{0.9}
	\newthread{usr}{User}
	\newinst[4]{app}{Application}
	\newinst[4]{svr}{Server}

	\begin{call}{usr}{Click Research Tab}{app}{Update Research Page}
		\begin{call}{app}{getAllResearch()}{svr}{researchList}
		\end{call}	
	\end{call}
	
	\begin{call}{usr}{Search for Research}{app}{Update Research Page}
	\end{call}	
	
	
	\begin{call}{usr}{Choose Research File}{app}{Display File Information}
		\begin{call}{app}{getFileMetadata()}{svr}{fileInfo}
		\end{call}	
	\end{call}
	
	\begin{call}{usr}{Click Download Button}{app}{Download File to Device}
		\begin{call}{app}{getFileURL()}{svr}{fileURL}
		\end{call}	
	\end{call}
	
	
\end{sequencediagram}
\end{center}



\clearpage
\section{Next Steps}
Although the core functionality of the API was finished, some of the projected features were left incomplete.

\subsection{Incomplete Requirements}
The features that are left to be completed are as follows.\\
\textbf{1. \hspace{8pt} Files cannot be uploaded or downloaded.}

Requirements: 13, 14, 15, 27 , 28, 36, 43 (MUST)\\
\textbf{2. \hspace{8pt} Research Files cannot be approved, submitted for review, deleted, or rejected.}

Requirements: 29, 30 , 31, 32, 33, 34, 35 (MUST)\\
\textbf{3. \hspace{8pt} Farmers cannot recover their password}

Requirements: 63 (MUST)\\ 
\textbf{4. \hspace{8pt} ARET cannot send notifications to Farmers.}

Requirements: 67, 68, 69, 70, 71, 72, 85, 89, 93 (SHOULD)\\
\textbf{5. \hspace{8pt} ARET cannot send surveys to Farmers.}

Requirements: 75, 76, 77, 78 (SHOULD)\\
\textbf{6. \hspace{8pt} System has no authentication.}

Requirements: 112, 113, 114 (SHOULD)\\
\textbf{7. \hspace{8pt} ARET cannot access feedback data from the Farmers.}

Requirements: 124, 125, 126 (COULD)

\subsection{Cost Calculations}
The incomplete requirements add up to a total of 48 hours of work.\\
Assuming a team of 5, working full time hours, the incomplete features would take 1 more iteration to complete.\\
As such, the cost to complete the project would be \$12,000.

\clearpage
\section{Individual Contributions}
\subsection{David DiMaria}
\textbf{Yes! And... Using Improv}\par


\clearpage
\subsection{Braydon Johnson}
\textbf{The Yes! And... Lab Demo}\par



\clearpage
\subsection{Josh Lemieux}
This course was overall an enjoyable experience. I feel like I learned a lot in respect to rounding out my abilities as a software developer.\par 
To start off I would like to say one thing that I really loved about this course. I really enjoyed the improv session at the beginning of the semester. I feel this is a great introduction to a class where you will be working together in groups throughout the entire semester. It opens up how you can feel to negative stimuli versus positive. Positive vibes bring out more creative ideas in people, and that’s the true benefit of the improv session.\par
	One thing I hated in the class was the extremely long delay in having the ability to code. Now I understand the importance of the class and that it is not about coding. BUT having said that I feel like going through the process of requirements gathering and designing prototypes and then not getting to see much of the project come to fruition is a little disappointing.\par 
	So the one change that I would suggest in the course is to actually change it to a 1.0 credit course. This would then let us accelerate the process of each design phase so that we can start implementing code earlier.


\clearpage
\subsection{Neivin Mathew}
\textbf{The Good, the Bad and the Ugly}\par
As the semester comes to a close, I have to admit that Systems Analysis and Design was one of my more enjoyable courses. I felt like I learned a lot of things that will help me in my career as a Software Engineer.\par
I especially enjoyed working in a group with some of my friends, and strangers who are now my friends. Since this was the first course that really enforced group projects on me, I was apprehensive at first, but soon came to enjoy it. Seeing different perspectives and forcing myself to listen to ideas that you would otherwise not consider broadened my point of view. Seeing a team effort come to fruition was one of the most satisfying experiences I have had in this course.\par
The one thing I hated about the course was the reverse classroom concept. While it seemed like a noble endeavor, I felt like going to class was pointless because of it. I could learn concepts on my own without attending the lectures, which seemed to defeat the very purpose of a lecture. In my opinion, the professor imparts knowledge that no textbook can provide, while the reverse is not true. Working on the assignments and projects during class time was an unnecessary bonus, since it was usually difficult to even make any progress on them in the middle of sixty people.\par
The one thing I would change for the project is the group evaluations segment. I felt like there wasn't enough accountability for each member of the group. A simple method of students logging hours or stating what they worked on for each report would help ease the burden off of members who find themselves doing extra work for the sake of their grade. It was way too easy to get a free ride on each assignment, since most people were hesitant to evaluate their friends poorly.\par
Overall, I had fun designing the system we were assigned in the course and will definitely use some of the skills I learned in the future.
		
	
\clearpage
\subsection{Like Zheng}
\textbf{Yes!And... Philosophy}\par


\end{document}