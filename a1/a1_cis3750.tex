\documentclass[12pt,letterpaper]{article}
\usepackage[utf8]{inputenc}
\usepackage{kpfonts}
\usepackage[T1]{fontenc}

\usepackage{titlesec}

\titlespacing*\section{0pt}{12pt plus 4pt minus 2pt}{0pt plus 2pt minus 2pt}
\titlespacing*\subsection{0pt}{12pt plus 4pt minus 2pt}{0pt plus 2pt minus 2pt}
\titlespacing*\subsubsection{0pt}{12pt plus 4pt minus 2pt}{0pt plus 2pt minus 2pt}

% Package for double spacing
\usepackage{setspace}

% Set 1.0 inch margins
\usepackage[margin=1.0in, headheight=15pt]{geometry}



% Use images and graphics
\usepackage{graphicx}
\usepackage{float}

% no item indent
\usepackage{enumitem}

% Use nicer headers
\usepackage{fancyhdr}
\pagestyle{fancy}
\renewcommand{\headrulewidth}{0pt}
\rhead{CIS3750 - Assignment 1}

\doublespacing

\title{Assignment 1}

\begin{document}
\begin{titlepage}
    \centering
    \vspace*{\baselineskip}
    \rule{\textwidth}{1.6pt}\vspace*{-\baselineskip}\vspace*{2pt}
    \rule{\textwidth}{0.4pt}\\[1.5\baselineskip]
    {\LARGE \textsc{An Initial Outline of a Software System to Improve Food Security in Malawi}}\\[\baselineskip]
	\rule{\textwidth}{0.4pt}\vspace*{-\baselineskip}\vspace{4pt}    
    \rule{\textwidth}{2pt}\\[2\baselineskip]
   
    \vspace*{5\baselineskip}
    \textsc{BY}\\[0.25\baselineskip]
    {\LARGE HANLON} \\
    
    \vspace*{\baselineskip}
    % List of authors in alphabetical order (by last name)
    {\textsc{David DiMaria \\ Braydon Johnson \\ Joshua Lemieux \\ Neivin Mathew \\ Like Zheng} \par}
    \vfill
    {\scshape September 30, 2016} \\
  \end{titlepage}
  
  
% Table of Contents (no page numbers on contents)
\pagenumbering{roman} %roman numerals for ToC
\tableofcontents
\lhead{} % remove default header from Contents page
\clearpage
\pagenumbering{arabic} %pagenumbering in arabic numbers
    
\section{Client Details}
Malawi is a country located in the warm heart of Africa, with a population of 16.4 million. Last year, there were 1 million people in Malawi facing a problem known as food insecurity. This year, that number will rise to 6.4 million. The economy of the country is based on agriculture. Most of the people in Malawi are farmers who grow tobacco for living. Due to the impact of climate change and the growing lack of food, the farmers need to transition to growing different crops. The problem is that many of the Malawians have never grown anything besides tobacco, and do not contain the knowledge to start growing other crops. This is the issue that ARET are trying to solve. \par

ARET (Agricultural Research and Extension Trust) is a research organization in Malawi. They are Malawi’s premier research institution, which are  responsible for conducting research and providing technical/extension services on tobacco. The Trust was established on September 1st, 1995 to foster development and information dissemination for Malawi’s tobacco industry. It amalgamated the services of two institutions, Tobacco Research Institute of Malawi (TRIM) and the Estate Extension Service Trust (EEST), who separately provided research and extension services respectively (reference 1). \par

ARET states their vision as “To be a leading regional centre of excellence in agricultural research and technology dissemination which promotes diversification in the agricultural sector.” (ARET Strategic Plan 2016-2021). 

\clearpage
\section{Team Details}
\subsection{Team Name and Logo}
The team name for the project is "Hanlon."\par
The name is inspired by the eponymous highway that runs through the city of Guelph, and signifies the team's ties to the University of Guelph, as well as the city of Guelph.\\

\begin{figure}[H]
	\centering	
	\includegraphics[height=2in]{img/hanlon-logo.png}
	%\caption{The team logo}
	\label{fig:kitten}
\end{figure}

\subsection{Team Members}
Hanlon is comprised of the following students:\\
1. \textbf{\hspace*{5pt} David DiMaria} - Project Manager\\
2. \textbf{\hspace*{5pt}Braydon Johnson} - Software Developer, User Interface Designer\\
3. \textbf{\hspace*{5pt}Joshua Lemieux} - Project Manager, User Interface Designer\\
4. \textbf{\hspace*{5pt}Neivin Mathew} - Software Developer, User Interface Designer\\
5. \textbf{\hspace*{5pt}Like Zheng} - Software Developer

\subsection{Team Roles}
\subsubsection*{Project Manager}
The Project Manager predicts potential problems that may arise during development, and plans tasks to ensure that the project is completed successfully and on time. This role involves the scheduling and unblocking of tasks. It may also involve some programming.

\subsubsection*{Software Developer}
The Developer is involved in all aspects of the software development process including research, design, coding, documentation and testing.

\subsubsection*{User Interface Designer}
The User Interface (UI) Designer role is to plan out and develop any user facing component of the system which includes the specific layout of screens, and improving the interaction between the customer and the product.

\subsection{Team Organization}
Hanlon will follow a static team structure. Each member will maintain their respective roles for the entire duration of the project. \par

Hanlon will use a democratic majority voting system for any decisions that need to be taken within the team. Each present member will be involved in voting, and possesses one vote per motion. A motion is passed when a simple majority is achieved. \par

In the event of a team member being unavailable, and a majority cannot be established, a motion can only be passed through unanimous consent.

\clearpage
\section{Project Goals and Users}
\subsection{Project Goals}

The goal of the project is to design and develop a system that allows the Agricultural Research and Extension Trust (ARET) of Malawi to provide agricultural information to the farmers of Malawi, and relay the data collected from the system back to ARET and its partners.

\subsection{Users}
1.\hspace*{5pt} \textbf{System Administrator -- } Maintains database and API.\\
2.\hspace*{5pt} \textbf{Researcher -- } Creates research materials and reviews extension materials. \\
3.\hspace*{5pt} \textbf{Extension Service --} Creates resources for farmers from research materials. \\
4.\hspace*{5pt} \textbf{Extension Officer -- } Acts as the liaison between the Researcher and Extension Agent \\
5.\hspace*{5pt} \textbf{Extension Agent -- } Acts as the liaison between the Extension Officer and the Farmer. \\
6.\hspace*{5pt} \textbf{Farmer -- } Views information resources , supplies farm data, and responds to surveys. \\
7.\hspace*{5pt} \textbf{Partner -- } Collaborates with ARET, and may access the database and web portal \\
8.\hspace*{5pt} \textbf{Public -- } Uses the system without a user account. Includes any individual that does not fit into the other user groups. \\
9.\hspace*{5pt} \textbf{System -- } Consists of the entire platform (web, mobile), database, and API.

\subsection{Project Organization}
Version control of the codebase will be done using Git. Trello will be used to keep track of sprints and other project management tasks. Communication within the team and between teams will be accomplished through Slack.\par
The database will be built with SQL, using MySQL or PostgreSQL. The API would be developed in Python on the Django REST Framework. The web portal and mobile application will be built concurrently using HTML5, CSS3, and JavaScript, while relying on the Adobe PhoneGap framework to create a hybrid platform.

\clearpage
\section{Requirements}
\subsection{Definitions}
The terms used in the requirements document are defined as follows:\\
1. \hspace*{5pt} \textbf{ARET -- } The Agricultural Research and Extension Trust of Malawi. \\
2. \hspace*{5pt} \textbf{SMS -- } Short Message Service. A service on cell phones that allows the exchange of short text messages.\\
3. \hspace*{5pt} \textbf{API -- } Application Programming Interface. A set of tools that allows applications to interact with each other.\\
4. \hspace*{5pt} \textbf{SQL -- } Structured Query Language. A language that allows the definition and manipulation of data.\\
5. \hspace*{5pt} \textbf{GUI -- } Graphical User Interface. A visual framework that enables easy interaction with an application.

\subsection{Requirements Table}
The table of requirements can be found as a .csv file attached to the report.

\clearpage
\section{Individual Contributions}
\subsection{Josh Lemieux}
\textbf{Collaboration is the New Competitive Advantage}\par
I’m going to be honest, after reading the articles provided and actually thinking about an answer to the question, I could not think of a single thing that I was passionate about that would truly help people. It was honestly a little bit of a shocker. It’s not that there was nothing I was passionate about, there would be plenty of technologies I would love to work on and companies I would love to work for. But none of them were really solving important issues, and that was a little depressing. So I decided to analyze my life and some of the things I may take for granted. I realized something in that second, I was sitting in a Starbucks drinking a ridiculously overpriced fancy latte. I thought\... here I am drinking a \$5 drink when millions of people in developing countries are just wishing for a little bit of fresh water.\par
So I wondered how much water could be bought with just \$5? I came across this article Here \par
Wow, less than \$5 per million liters of water removed. So, for the price of my one \$5 latte, Nestle could give a liter of fresh water to a million people. That is simply astonishing. I don’t think I really need to explain how important it is to have fresh water, but if you’re up for a challenge just avoid drinking anything for a full twenty four hours, you will soon realize how bad being dehydrated can be. \par
Now what would I do if I had practically infinite funds? I would simply buy a few companies like Nestle and stop worrying about having the perfect profit margins and start worrying about my fellow mankind. The most depressing part is, even with the changes I would make the company would likely still be profitable. So why not now?


\clearpage
\subsection{David DiMaria}
\textbf{Collaboration is the New Competitive Advantage}

After reading the articles provided, there was a lot to think about in regards to how I am going to use my degree after I graduate. Obviously I’d like to be employed by a software company, and make good money. However, I hadn’t given much thought to where I would want to work within computer science, and whether or not it was important that I’m passionate about what I’m working on professionally. If I could use my skills towards some world challenge, it would be the right to knowledge, and specifically the right to free internet. To elaborate on that, I grew up in the age where by the time I was able to type, search engines were extremely popular and any information that I wanted was right at my fingertips. I could know anything at any moment, uncensored, as long as I had a stable internet connection. I could learn about anything I want, immerse myself in online communities, share information with my peers and find like minded individuals from all around the world.\par

This is a challenge since a large portion of the world doesn’t have internet access, and many that do have portions censored based on which country they're from. This is something that people should care about because if impoverished people can learn anything at any moment, they could be saved from governments trying to suppress them physically and intellectually. The reason why governments will try and censor things is because of the amount of political unrest it would cause given there people had all the information. The power that information could have on a population is unfathomable, because they could learn the truth about how their government is perceived in the eyes of the rest of the world.\par

If I could choose a group of people to help me achieve this goal, it would be Google, Facebook, and telecommunications companies. The power of internet infrastructure at companies like Google and Facebook is large enough that they could develop a worldwide satellite network with telecommunications companies that delivers internet to everyone.

\end{document}