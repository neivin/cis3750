\documentclass[12pt,letterpaper]{article}
\usepackage[utf8]{inputenc}
\usepackage{kpfonts}
\usepackage[T1]{fontenc}

% custom titles
\usepackage{titlesec}

% fix broken title numbering with 2016 titlesec update
\usepackage{etoolbox}

\makeatletter
\patchcmd{\ttlh@hang}{\parindent\z@}{\parindent\z@\leavevmode}{}{}
\patchcmd{\ttlh@hang}{\noindent}{}{}{}
\makeatother
%%%

\titlespacing*\section{0pt}{12pt plus 4pt minus 2pt}{0pt plus 2pt minus 2pt}
\titlespacing*\subsection{0pt}{12pt plus 4pt minus 2pt}{0pt plus 2pt minus 2pt}
\titlespacing*\subsubsection{0pt}{12pt plus 4pt minus 2pt}{0pt plus 2pt minus 2pt}

% Package for double spacing
\usepackage{setspace}
\usepackage{ragged2e}

% Set 1.0 inch margins
\usepackage[margin=1.0in, headheight=15pt]{geometry}
\usepackage{enumitem}

% Use images and graphics
\usepackage{graphicx}
\usepackage{float}

% Use nicer headers
\usepackage{fancyhdr}
\pagestyle{fancy}
\renewcommand{\headrulewidth}{0pt}
\rhead{CIS3750 Final Report}

% sections should be indexed with alphabets
\renewcommand{\thesection}{\Alph{section}}

%double spaced lines in the whole document
\doublespacing

\title{Final Report}

\begin{document}
\begin{titlepage}
    \centering
    \vspace*{\baselineskip}
    \rule{\textwidth}{1.6pt}\vspace*{-\baselineskip}\vspace*{2pt}
    \rule{\textwidth}{0.4pt}\\[1.5\baselineskip]
    {\LARGE \textsc{Reflections on a Software Prototype}}\\[\baselineskip]
	\rule{\textwidth}{0.4pt}\vspace*{-\baselineskip}\vspace{4pt}    
    \rule{\textwidth}{2pt}\\[2\baselineskip]
   
    \vspace*{5\baselineskip}
    \textsc{BY}\\[0.25\baselineskip]
    {\LARGE HANLON} \\
    
    \vspace*{\baselineskip}
    % List of authors in alphabetical order (by last name)
    {\textsc{David DiMaria \\ Braydon Johnson \\ Joshua Lemieux \\ Neivin Mathew \\ Like Zheng} \par}
    \vfill
    {\scshape December 5, 2016} \\
  \end{titlepage}
  
  
% Table of Contents (no page numbers on contents)
\pagenumbering{roman} %roman numerals for ToC
\tableofcontents
\lhead{} % remove default header from Contents page
\clearpage
\pagenumbering{arabic} %pagenumbering in arabic numbers
    
\clearpage
\section{Video}
The Final Prototyping Session was completed on November 23, 2016 during the scheduled lab time.


\clearpage
\section{Looking to the Future}
Throughout the prototyping session, the participants made a number of suggestions about improvements that they would like to see to some features of the application. Some also wished that there were additional features for certain parts of the application.\par	
In a future version of our API we would like to add a system to allow the ARET employees to send messages/notifications to farmers. We would create a database table to upload messages that are specific to a set of farmers (region, crop they are growing, etc.), and also implement a system to send push notifications to the mobile application.\par
One of the participants noticed that we had the option to upload and download research documents, which had not been implemented for the prototyping session. Since we did not get to implement hosting and retrieving files in our API, we would need to allow certain users to be able to upload files to a web server and have a database table containing information on the file (size, link, author).\par
Another participant had some concerns about the security of our system. Currently the only security measures implemented in our system is that of hashing account passwords on the server. However, this is only done once the API has received the plain text passwords. In the future we would decide on a hashing algorithm across all the teams, and the hashing and authentication would be done on both ends. Additionally, there is no security around the API itself, so anyone with the URL can access it. In the future, we would implement a token authentication to prevent unauthorized use of the API.\par

\clearpage
\section{Individual Contributions}
\subsection{David DiMaria}
\textbf{Yes! And... Using Improv}\par
Considering I was the facilitator, and I chose to go without a script, I had to use some of the improv skills that we learned in class from The Making Box. First of all I had to make eye contact when explaining the purpose of our app and what paper prototyping is. This was a helpful skill because I needed to make sure they understood how to paper prototype, and maintaining eye contact makes it easier for them to listen to what I was saying.\par
One of the skills that I actually couldn’t use was the Yes, And… skill. The reason for this is when a participant asked for clarification on a use case, I wasn’t allowed to lead them in the right direction. As much as I wanted to lead them in the right direction with a new idea, it would have made it more difficult to determine if the system was intuitive to them. Also I explicitly wasn’t allowed to give suggestions in the paper prototype session, so I had to abandon the Yes, And… tactic for the morning. 


\clearpage
\subsection{Braydon Johnson}
\textbf{The Yes! And... Lab Demo}\par
During the lab demo I did not use any of the skills from the Yes! And… philosophy, this was because I was a note taker and did not interact with the participants at all. Honestly, I don’t believe the skills would have been that useful, previously when I talked about it they were helpful because they gave us the tools to build on other people’s ideas rather than shut them down in place of our own ideas.\par 
In the prototyping session except for the introduction it was mostly watching and listening to the client, which certainly limited the options of techniques to use especially for those of us who had no real reason to interact with the participant. While there are some tools that promote our listening skills I previously stated that I wasn’t very fond of First Letter Last Letter as it made it too hard to focus on what they were saying as a whole because I was too busy listening for the end of their sentence.


\clearpage
\subsection{Josh Lemieux}
\textbf{Yes!And...Lab Demo}\par
For our paper prototyping session, I personally did not use any of the techniques as I was the Human Computer (Beep Boop). I was not allowed to communicate with the client. I was only there to provide smooth transitions between screens following the client's’ input. There were times where my group members used a little bit of knowledge gained during the prototyping session though.\par
	The last client had some rather odd suggestions that were not typical in mobile applications. Rather than immediately shutting the idea down we thought about how that would be implemented and work in our app. It turns out that they just didn’t work out with our app, but that’s okay! By taking that time to review a suggestion that initially seems to not fit with our application, we may come to see that an odd suggestion could add benefit to our application.\par
      	A lot that we learned in the improv session is not exactly applicable to the paper prototyping session. This is because you are supposed to keep your interactions with the client to be as limited as possible. This is to get their “natural” reaction to using the application. I can see us using these techniques for our wire framing session when we are communicating more directly with the client.


\clearpage
\subsection{Neivin Mathew}
\textbf{Improv and Open-Mindedness}\par
It was rather difficult to completely apply the techniques I learned during the Improv session during the final prototyping session, mostly because the participants I spoke to merely asked me questions about our API, rather than having me guide them through an application.\par
However, the principles behind the "Yes!And..." philosophy did help me analyze the prototype more critically.\par 
It is easy to get invested in something when you have worked on it for a long time. You begin to look at your brainchild through rose coloured glasses and often remain oblivious to glaring errors in design. It's very hard to even accept criticism for your work, especially after you have spent hundreds of hours perfecting it. \par
However, the methods I learned in the Improv session definitely helped me keep an open mind about our design and forced me to be more flexible about possible changes. Accepting every change suggested by the user really opened my eyes to some design flaws that I would not have seen otherwise. For example, I was under the impression that our API responses were served appropriately to both the mobile and web teams. One of our participants suggested that each team should get a more tailored response to suit their needs. This design choice made a lot of sense, and I would not have considered this suggestion if it hadn't been for the lessons I learned from the Improv session. Remembering the methodology of "Yes!And..." really helped me realize the importance of considering different perspectives in design, especially that of a potential user. 

		
	
\clearpage
\subsection{Like Zheng}
\textbf{Yes!And... Philosophy}\par
During the prototyping session, my role was that of observer and note taker. I was responsible for observing the user's behaviour, taking notes about how the system performs, and how the user responds as well.\par
Thus, it was not my job to talk to the participant. However, our facilitator was responsible for that. I noticed that the way our facilitator communicated with participant was much more listening than talking. This is important to minimize any influence our words would have on the participant, and allow them to give us their unbiased opinion. They can now give us a true reflection of what they are thinking.\par 
The “Yes! And… philosophy was not useful during prototyping session. It is a tool for changing ideas, and in prototyping session, we were listening to the user's ideas for most of the time. Thus, I did not find the technique helpful for the prototyping session. 

\end{document}